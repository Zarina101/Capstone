\PassOptionsToPackage{unicode=true}{hyperref} % options for packages loaded elsewhere
\PassOptionsToPackage{hyphens}{url}
%
\documentclass[]{article}
\usepackage{lmodern}
\usepackage{amssymb,amsmath}
\usepackage{ifxetex,ifluatex}
\usepackage{fixltx2e} % provides \textsubscript
\ifnum 0\ifxetex 1\fi\ifluatex 1\fi=0 % if pdftex
  \usepackage[T1]{fontenc}
  \usepackage[utf8]{inputenc}
  \usepackage{textcomp} % provides euro and other symbols
\else % if luatex or xelatex
  \usepackage{unicode-math}
  \defaultfontfeatures{Ligatures=TeX,Scale=MatchLowercase}
\fi
% use upquote if available, for straight quotes in verbatim environments
\IfFileExists{upquote.sty}{\usepackage{upquote}}{}
% use microtype if available
\IfFileExists{microtype.sty}{%
\usepackage[]{microtype}
\UseMicrotypeSet[protrusion]{basicmath} % disable protrusion for tt fonts
}{}
\IfFileExists{parskip.sty}{%
\usepackage{parskip}
}{% else
\setlength{\parindent}{0pt}
\setlength{\parskip}{6pt plus 2pt minus 1pt}
}
\usepackage{hyperref}
\hypersetup{
            pdftitle={PartB},
            pdfauthor={Ulita Gilbert},
            pdfborder={0 0 0},
            breaklinks=true}
\urlstyle{same}  % don't use monospace font for urls
\usepackage[margin=1in]{geometry}
\usepackage{color}
\usepackage{fancyvrb}
\newcommand{\VerbBar}{|}
\newcommand{\VERB}{\Verb[commandchars=\\\{\}]}
\DefineVerbatimEnvironment{Highlighting}{Verbatim}{commandchars=\\\{\}}
% Add ',fontsize=\small' for more characters per line
\usepackage{framed}
\definecolor{shadecolor}{RGB}{248,248,248}
\newenvironment{Shaded}{\begin{snugshade}}{\end{snugshade}}
\newcommand{\AlertTok}[1]{\textcolor[rgb]{0.94,0.16,0.16}{#1}}
\newcommand{\AnnotationTok}[1]{\textcolor[rgb]{0.56,0.35,0.01}{\textbf{\textit{#1}}}}
\newcommand{\AttributeTok}[1]{\textcolor[rgb]{0.77,0.63,0.00}{#1}}
\newcommand{\BaseNTok}[1]{\textcolor[rgb]{0.00,0.00,0.81}{#1}}
\newcommand{\BuiltInTok}[1]{#1}
\newcommand{\CharTok}[1]{\textcolor[rgb]{0.31,0.60,0.02}{#1}}
\newcommand{\CommentTok}[1]{\textcolor[rgb]{0.56,0.35,0.01}{\textit{#1}}}
\newcommand{\CommentVarTok}[1]{\textcolor[rgb]{0.56,0.35,0.01}{\textbf{\textit{#1}}}}
\newcommand{\ConstantTok}[1]{\textcolor[rgb]{0.00,0.00,0.00}{#1}}
\newcommand{\ControlFlowTok}[1]{\textcolor[rgb]{0.13,0.29,0.53}{\textbf{#1}}}
\newcommand{\DataTypeTok}[1]{\textcolor[rgb]{0.13,0.29,0.53}{#1}}
\newcommand{\DecValTok}[1]{\textcolor[rgb]{0.00,0.00,0.81}{#1}}
\newcommand{\DocumentationTok}[1]{\textcolor[rgb]{0.56,0.35,0.01}{\textbf{\textit{#1}}}}
\newcommand{\ErrorTok}[1]{\textcolor[rgb]{0.64,0.00,0.00}{\textbf{#1}}}
\newcommand{\ExtensionTok}[1]{#1}
\newcommand{\FloatTok}[1]{\textcolor[rgb]{0.00,0.00,0.81}{#1}}
\newcommand{\FunctionTok}[1]{\textcolor[rgb]{0.00,0.00,0.00}{#1}}
\newcommand{\ImportTok}[1]{#1}
\newcommand{\InformationTok}[1]{\textcolor[rgb]{0.56,0.35,0.01}{\textbf{\textit{#1}}}}
\newcommand{\KeywordTok}[1]{\textcolor[rgb]{0.13,0.29,0.53}{\textbf{#1}}}
\newcommand{\NormalTok}[1]{#1}
\newcommand{\OperatorTok}[1]{\textcolor[rgb]{0.81,0.36,0.00}{\textbf{#1}}}
\newcommand{\OtherTok}[1]{\textcolor[rgb]{0.56,0.35,0.01}{#1}}
\newcommand{\PreprocessorTok}[1]{\textcolor[rgb]{0.56,0.35,0.01}{\textit{#1}}}
\newcommand{\RegionMarkerTok}[1]{#1}
\newcommand{\SpecialCharTok}[1]{\textcolor[rgb]{0.00,0.00,0.00}{#1}}
\newcommand{\SpecialStringTok}[1]{\textcolor[rgb]{0.31,0.60,0.02}{#1}}
\newcommand{\StringTok}[1]{\textcolor[rgb]{0.31,0.60,0.02}{#1}}
\newcommand{\VariableTok}[1]{\textcolor[rgb]{0.00,0.00,0.00}{#1}}
\newcommand{\VerbatimStringTok}[1]{\textcolor[rgb]{0.31,0.60,0.02}{#1}}
\newcommand{\WarningTok}[1]{\textcolor[rgb]{0.56,0.35,0.01}{\textbf{\textit{#1}}}}
\usepackage{longtable,booktabs}
% Fix footnotes in tables (requires footnote package)
\IfFileExists{footnote.sty}{\usepackage{footnote}\makesavenoteenv{longtable}}{}
\usepackage{graphicx,grffile}
\makeatletter
\def\maxwidth{\ifdim\Gin@nat@width>\linewidth\linewidth\else\Gin@nat@width\fi}
\def\maxheight{\ifdim\Gin@nat@height>\textheight\textheight\else\Gin@nat@height\fi}
\makeatother
% Scale images if necessary, so that they will not overflow the page
% margins by default, and it is still possible to overwrite the defaults
% using explicit options in \includegraphics[width, height, ...]{}
\setkeys{Gin}{width=\maxwidth,height=\maxheight,keepaspectratio}
\setlength{\emergencystretch}{3em}  % prevent overfull lines
\providecommand{\tightlist}{%
  \setlength{\itemsep}{0pt}\setlength{\parskip}{0pt}}
\setcounter{secnumdepth}{0}
% Redefines (sub)paragraphs to behave more like sections
\ifx\paragraph\undefined\else
\let\oldparagraph\paragraph
\renewcommand{\paragraph}[1]{\oldparagraph{#1}\mbox{}}
\fi
\ifx\subparagraph\undefined\else
\let\oldsubparagraph\subparagraph
\renewcommand{\subparagraph}[1]{\oldsubparagraph{#1}\mbox{}}
\fi

% set default figure placement to htbp
\makeatletter
\def\fps@figure{htbp}
\makeatother


\title{PartB}
\author{Ulita Gilbert}
\date{05/03/2021}

\begin{document}
\maketitle

{
\setcounter{tocdepth}{2}
\tableofcontents
}
\hypertarget{data-dictionary}{%
\section{Data Dictionary}\label{data-dictionary}}

For the purposes of this EDA, the following variables and definitions
are important.

The population in this dataset is the incoming cohort of students in
Fall of 2019 and 2020. These students are first time degree or
certificate seekers and have no prior tertiary education. They may have
earned AP credits in highschool.

\textbf{Fall2019} refers to the incoming freshman cohort in Fall2019.
This is term year 2020.\\
\textbf{Fall2020} refers to the incoming freshman cohort in Fall2020.
This is term year 2021.

\textbf{Variables of Interest:}\\
\textbf{hours\_earned:} refers to credit hours the student has earned in
their first Fall semester ( this can include credits earned in Summer
school second session- Summer 1 and AP credits earned in highschool).
\textbf{hours\_attempted:} refers to credit hours the student has
attempted in their first Fall semester ( this may include credits
attempted in Summerschool second session - Summer 1) and hours spent on
developmental non credit courses.

\textbf{full\_part:} is the student full-time (FT) or part-time (PT)

\textbf{major:} degree programme student is registered for or
certificate\&LR ( letter of recommendation.) All certificates and
letters of recommendations have been grouped together.

\textbf{hours\_earned\_rate:} Ratio of hours\_earned/hours\_attempted

\hypertarget{import-data}{%
\section{Import Data}\label{import-data}}

\begin{Shaded}
\begin{Highlighting}[]
\CommentTok{#import data for Fall 2016 to Fall 2020}
\CommentTok{# set directory}
\KeywordTok{setwd}\NormalTok{(}\StringTok{'/Users/ulita/Documents/CAPSTONE205/Capstone/Capstone_Data/Raw_Data'}\NormalTok{)}

\NormalTok{initial_df_}\DecValTok{2019}\NormalTok{<-}\KeywordTok{read_xlsx}\NormalTok{(}\StringTok{"FTEIC F19.xlsx"}\NormalTok{)}
\NormalTok{initial_df_}\DecValTok{2020}\NormalTok{<-}\KeywordTok{read_xlsx}\NormalTok{(}\StringTok{"FTEIC F20.xlsx"}\NormalTok{)}
\end{Highlighting}
\end{Shaded}

\hypertarget{data-wrangling}{%
\section{Data Wrangling}\label{data-wrangling}}

\hypertarget{create-dataframe}{%
\section{Create Dataframe}\label{create-dataframe}}

\begin{Shaded}
\begin{Highlighting}[]
 \CommentTok{#create dataframe to compare credits in Fall 2019 and Fall 2020 :    df_credits }
\NormalTok{df_credits<-}\StringTok{ }\KeywordTok{bind_rows}\NormalTok{(initial_df_}\DecValTok{2019}\NormalTok{,initial_df_}\DecValTok{2020}\NormalTok{)}

\KeywordTok{head}\NormalTok{(df_credits)}
\end{Highlighting}
\end{Shaded}

\begin{verbatim}
## # A tibble: 6 x 50
##   `U-Number` Sex   IPEDS_RACE_DESC `Age at Start` HIGH_SCHOOL FULL_PART CITY 
##        <dbl> <chr> <chr>           <chr>          <chr>       <chr>     <chr>
## 1   20190001 F     Hispanic        18 - 20        Out of Sta~ PT        Gait~
## 2   20190002 M     Hispanic        18 - 20        Walter Joh~ FT        Rock~
## 3   20190003 F     Multi-Race      18 - 20        Wheaton Hi~ PT        Gait~
## 4   20190004 M     Asian           18 - 20        Thomas Spr~ PT        Gait~
## 5   20190005 M     Hispanic        18 - 20        FOREIGN / ~ FT        Chev~
## 6   20190006 F     Hispanic        18 - 20        Springbroo~ FT        Silv~
## # ... with 43 more variables: STAT_CODE <chr>, ZIP <dbl>, PELL_GRANT <chr>,
## #   CAMP_CODE <chr>, Major <chr>, HOURS_ATTEMPTED <chr>, HOURS_EARNED <chr>,
## #   MC_GPA <dbl>, PASS_ENGL <chr>, PASS_MATH <chr>, term_year <chr>,
## #   TERM_YEAR1_10 <chr>, TERM_YEAR1_20 <chr>, TERM_YEAR1_24 <chr>,
## #   TERM_YEAR1_25 <chr>, TERM_YEAR1_30 <chr>, TERM_YEAR1_40 <chr>,
## #   TERM_YEAR2_10 <chr>, TERM_YEAR2_20 <chr>, TERM_YEAR2_24 <chr>,
## #   TERM_YEAR2_25 <chr>, TERM_YEAR2_30 <chr>, TERM_YEAR2_40 <chr>,
## #   TERM_YEAR3_10 <chr>, TERM_YEAR3_20 <chr>, TERM_YEAR3_24 <chr>,
## #   TERM_YEAR3_25 <chr>, TERM_YEAR3_30 <chr>, TERM_YEAR3_40 <chr>,
## #   TERM_YEAR4_10 <chr>, TERM_YEAR4_20 <chr>, TERM_YEAR4_24 <chr>,
## #   TERM_YEAR4_25 <chr>, TERM_YEAR4_30 <chr>, TERM_YEAR4_40 <chr>,
## #   TERM_YEAR5_10 <chr>, TERM_YEAR5_20 <chr>, TERM_YEAR5_24 <chr>,
## #   TERM_YEAR5_25 <chr>, TERM_YEAR5_30 <chr>, first_grad_date <lgl>,
## #   first_degree <lgl>, TERM_YEAR5_40 <chr>
\end{verbatim}

\hypertarget{select-columns-of-interest-format-names}{%
\subsection{Select Columns of Interest \& Format
Names}\label{select-columns-of-interest-format-names}}

\begin{Shaded}
\begin{Highlighting}[]
\CommentTok{#clean names using janitor package}
\NormalTok{df_credits<-df_credits }\OperatorTok
\StringTok{             }\KeywordTok{clean_names}\NormalTok{()}

\CommentTok{#drop columns not interested in}

\NormalTok{df_credits<-df_credits }\OperatorTok
\StringTok{             }\KeywordTok{select}\NormalTok{(}\KeywordTok{c}\NormalTok{(}\DecValTok{1}\OperatorTok{:}\DecValTok{18}\NormalTok{))}

\CommentTok{# rename columns to make it user friendly}
\NormalTok{df_credits<-df_credits }\OperatorTok
\StringTok{            }\KeywordTok{rename}\NormalTok{(}\StringTok{"age"}\NormalTok{=}\StringTok{ "age_at_start"}\NormalTok{, }\StringTok{"race"}\NormalTok{=}\StringTok{ "ipeds_race_desc"}\NormalTok{)}
\end{Highlighting}
\end{Shaded}

\hypertarget{overview-of-data}{%
\subsection{Overview of Data}\label{overview-of-data}}

\begin{Shaded}
\begin{Highlighting}[]
\CommentTok{# summary statistics of dataframing using function skimr creates a table of summary statistics of all variables in the dataframe}
\KeywordTok{skim}\NormalTok{(df_credits)}
\end{Highlighting}
\end{Shaded}

\begin{longtable}[]{@{}ll@{}}
\caption{Data summary}\tabularnewline
\toprule
\endhead
Name & df\_credits\tabularnewline
Number of rows & 7519\tabularnewline
Number of columns & 18\tabularnewline
\_\_\_\_\_\_\_\_\_\_\_\_\_\_\_\_\_\_\_\_\_\_\_ &\tabularnewline
Column type frequency: &\tabularnewline
character & 15\tabularnewline
numeric & 3\tabularnewline
\_\_\_\_\_\_\_\_\_\_\_\_\_\_\_\_\_\_\_\_\_\_\_\_ &\tabularnewline
Group variables & None\tabularnewline
\bottomrule
\end{longtable}

\textbf{Variable type: character}

\begin{longtable}[]{@{}lrrrrrrr@{}}
\toprule
skim\_variable & n\_missing & complete\_rate & min & max & empty &
n\_unique & whitespace\tabularnewline
\midrule
\endhead
sex & 0 & 1.00 & 1 & 1 & 0 & 4 & 0\tabularnewline
race & 0 & 1.00 & 5 & 22 & 0 & 9 & 0\tabularnewline
age & 0 & 1.00 & 4 & 7 & 0 & 5 & 0\tabularnewline
high\_school & 0 & 1.00 & 7 & 30 & 0 & 172 & 0\tabularnewline
full\_part & 0 & 1.00 & 2 & 2 & 0 & 2 & 0\tabularnewline
city & 21 & 1.00 & 5 & 19 & 0 & 132 & 0\tabularnewline
stat\_code & 21 & 1.00 & 2 & 2 & 0 & 16 & 0\tabularnewline
pell\_grant & 0 & 1.00 & 1 & 1 & 0 & 2 & 0\tabularnewline
camp\_code & 154 & 0.98 & 1 & 1 & 0 & 6 & 0\tabularnewline
major & 33 & 1.00 & 1 & 61 & 0 & 35 & 0\tabularnewline
hours\_attempted & 87 & 0.99 & 2 & 4 & 0 & 55 & 0\tabularnewline
hours\_earned & 87 & 0.99 & 2 & 4 & 0 & 56 & 0\tabularnewline
pass\_engl & 0 & 1.00 & 1 & 1 & 0 & 2 & 0\tabularnewline
pass\_math & 0 & 1.00 & 1 & 1 & 0 & 2 & 0\tabularnewline
term\_year & 0 & 1.00 & 4 & 4 & 0 & 2 & 0\tabularnewline
\bottomrule
\end{longtable}

\textbf{Variable type: numeric}

\begin{longtable}[]{@{}lrrrrrrrrrl@{}}
\toprule
\begin{minipage}[b]{0.09\columnwidth}\raggedright
skim\_variable\strut
\end{minipage} & \begin{minipage}[b]{0.06\columnwidth}\raggedleft
n\_missing\strut
\end{minipage} & \begin{minipage}[b]{0.09\columnwidth}\raggedleft
complete\_rate\strut
\end{minipage} & \begin{minipage}[b]{0.07\columnwidth}\raggedleft
mean\strut
\end{minipage} & \begin{minipage}[b]{0.05\columnwidth}\raggedleft
sd\strut
\end{minipage} & \begin{minipage}[b]{0.06\columnwidth}\raggedleft
p0\strut
\end{minipage} & \begin{minipage}[b]{0.07\columnwidth}\raggedleft
p25\strut
\end{minipage} & \begin{minipage}[b]{0.07\columnwidth}\raggedleft
p50\strut
\end{minipage} & \begin{minipage}[b]{0.07\columnwidth}\raggedleft
p75\strut
\end{minipage} & \begin{minipage}[b]{0.06\columnwidth}\raggedleft
p100\strut
\end{minipage} & \begin{minipage}[b]{0.04\columnwidth}\raggedright
hist\strut
\end{minipage}\tabularnewline
\midrule
\endhead
\begin{minipage}[t]{0.09\columnwidth}\raggedright
u\_number\strut
\end{minipage} & \begin{minipage}[t]{0.06\columnwidth}\raggedleft
0\strut
\end{minipage} & \begin{minipage}[t]{0.09\columnwidth}\raggedleft
1.00\strut
\end{minipage} & \begin{minipage}[t]{0.07\columnwidth}\raggedleft
20196656.07\strut
\end{minipage} & \begin{minipage}[t]{0.05\columnwidth}\raggedleft
5028.73\strut
\end{minipage} & \begin{minipage}[t]{0.06\columnwidth}\raggedleft
20190001\strut
\end{minipage} & \begin{minipage}[t]{0.07\columnwidth}\raggedleft
20191880.5\strut
\end{minipage} & \begin{minipage}[t]{0.07\columnwidth}\raggedleft
20193760.0\strut
\end{minipage} & \begin{minipage}[t]{0.07\columnwidth}\raggedleft
20201708.5\strut
\end{minipage} & \begin{minipage}[t]{0.06\columnwidth}\raggedleft
20203588\strut
\end{minipage} & \begin{minipage}[t]{0.04\columnwidth}\raggedright
▇▃▁▂▇\strut
\end{minipage}\tabularnewline
\begin{minipage}[t]{0.09\columnwidth}\raggedright
zip\strut
\end{minipage} & \begin{minipage}[t]{0.06\columnwidth}\raggedleft
21\strut
\end{minipage} & \begin{minipage}[t]{0.09\columnwidth}\raggedleft
1.00\strut
\end{minipage} & \begin{minipage}[t]{0.07\columnwidth}\raggedleft
20884.18\strut
\end{minipage} & \begin{minipage}[t]{0.05\columnwidth}\raggedleft
1518.62\strut
\end{minipage} & \begin{minipage}[t]{0.06\columnwidth}\raggedleft
1460\strut
\end{minipage} & \begin{minipage}[t]{0.07\columnwidth}\raggedleft
20853.0\strut
\end{minipage} & \begin{minipage}[t]{0.07\columnwidth}\raggedleft
20877.0\strut
\end{minipage} & \begin{minipage}[t]{0.07\columnwidth}\raggedleft
20902.0\strut
\end{minipage} & \begin{minipage}[t]{0.06\columnwidth}\raggedleft
94025\strut
\end{minipage} & \begin{minipage}[t]{0.04\columnwidth}\raggedright
▁▇▁▁▁\strut
\end{minipage}\tabularnewline
\begin{minipage}[t]{0.09\columnwidth}\raggedright
mc\_gpa\strut
\end{minipage} & \begin{minipage}[t]{0.06\columnwidth}\raggedleft
87\strut
\end{minipage} & \begin{minipage}[t]{0.09\columnwidth}\raggedleft
0.99\strut
\end{minipage} & \begin{minipage}[t]{0.07\columnwidth}\raggedleft
2.16\strut
\end{minipage} & \begin{minipage}[t]{0.05\columnwidth}\raggedleft
1.48\strut
\end{minipage} & \begin{minipage}[t]{0.06\columnwidth}\raggedleft
0\strut
\end{minipage} & \begin{minipage}[t]{0.07\columnwidth}\raggedleft
0.5\strut
\end{minipage} & \begin{minipage}[t]{0.07\columnwidth}\raggedleft
2.5\strut
\end{minipage} & \begin{minipage}[t]{0.07\columnwidth}\raggedleft
3.5\strut
\end{minipage} & \begin{minipage}[t]{0.06\columnwidth}\raggedleft
4\strut
\end{minipage} & \begin{minipage}[t]{0.04\columnwidth}\raggedright
▆▂▃▅▇\strut
\end{minipage}\tabularnewline
\bottomrule
\end{longtable}

There are 18 variables and 7519 rows of data. Each row of data
represents 1 student. There are 3931 rows of data for Fall2019 and 3588
rows of data for Fall2020. Need to change the data types for
hours\_attempted and hours\_earned to numeric. u\_number should be a
charater. Most of the missing values are in camp\_code. For the purposes
of this exercise, this variable is not important. There are also 87
missing values in hours attempted and hours earned. As these two are the
variables of interest and GPA are of interest, I will drop the rows in
these variables with missing values.

\hypertarget{change-datatypes}{%
\subsection{Change Datatypes}\label{change-datatypes}}

\begin{Shaded}
\begin{Highlighting}[]
\CommentTok{# Change datatypes}
\NormalTok{df_credits}\OperatorTok{$}\NormalTok{hours_attempted<-}\StringTok{ }\KeywordTok{as.integer}\NormalTok{(df_credits}\OperatorTok{$}\NormalTok{hours_attempted)}
\NormalTok{df_credits}\OperatorTok{$}\NormalTok{hours_earned<-}\StringTok{ }\KeywordTok{as.integer}\NormalTok{(df_credits}\OperatorTok{$}\NormalTok{hours_earned)}
\NormalTok{df_credits}\OperatorTok{$}\NormalTok{mc_gpa<-}\StringTok{ }\KeywordTok{as.double}\NormalTok{(df_credits}\OperatorTok{$}\NormalTok{mc_gpa)}
\NormalTok{df_credits}\OperatorTok{$}\NormalTok{u_number<-}\StringTok{ }\KeywordTok{as.character}\NormalTok{(df_credits}\OperatorTok{$}\NormalTok{u_number)}
\NormalTok{df_credits}\OperatorTok{$}\NormalTok{zip<-}\StringTok{ }\KeywordTok{as.character}\NormalTok{(df_credits}\OperatorTok{$}\NormalTok{zip)}

\CommentTok{#drop missing values in hours_attempted, hours_earned and mc_gap}

\NormalTok{df_credits <-}\StringTok{ }\NormalTok{df_credits }\OperatorTok
\StringTok{              }\KeywordTok{drop_na}\NormalTok{(hours_attempted,hours_earned)}
\end{Highlighting}
\end{Shaded}

Summarise data again

\begin{Shaded}
\begin{Highlighting}[]
\KeywordTok{skim}\NormalTok{(df_credits)}
\end{Highlighting}
\end{Shaded}

\begin{longtable}[]{@{}ll@{}}
\caption{Data summary}\tabularnewline
\toprule
\endhead
Name & df\_credits\tabularnewline
Number of rows & 7432\tabularnewline
Number of columns & 18\tabularnewline
\_\_\_\_\_\_\_\_\_\_\_\_\_\_\_\_\_\_\_\_\_\_\_ &\tabularnewline
Column type frequency: &\tabularnewline
character & 15\tabularnewline
numeric & 3\tabularnewline
\_\_\_\_\_\_\_\_\_\_\_\_\_\_\_\_\_\_\_\_\_\_\_\_ &\tabularnewline
Group variables & None\tabularnewline
\bottomrule
\end{longtable}

\textbf{Variable type: character}

\begin{longtable}[]{@{}lrrrrrrr@{}}
\toprule
skim\_variable & n\_missing & complete\_rate & min & max & empty &
n\_unique & whitespace\tabularnewline
\midrule
\endhead
u\_number & 0 & 1.00 & 8 & 8 & 0 & 7432 & 0\tabularnewline
sex & 0 & 1.00 & 1 & 1 & 0 & 4 & 0\tabularnewline
race & 0 & 1.00 & 5 & 22 & 0 & 9 & 0\tabularnewline
age & 0 & 1.00 & 4 & 7 & 0 & 5 & 0\tabularnewline
high\_school & 0 & 1.00 & 7 & 30 & 0 & 171 & 0\tabularnewline
full\_part & 0 & 1.00 & 2 & 2 & 0 & 2 & 0\tabularnewline
city & 20 & 1.00 & 5 & 19 & 0 & 131 & 0\tabularnewline
stat\_code & 20 & 1.00 & 2 & 2 & 0 & 16 & 0\tabularnewline
zip & 20 & 1.00 & 4 & 5 & 0 & 166 & 0\tabularnewline
pell\_grant & 0 & 1.00 & 1 & 1 & 0 & 2 & 0\tabularnewline
camp\_code & 152 & 0.98 & 1 & 1 & 0 & 6 & 0\tabularnewline
major & 33 & 1.00 & 1 & 61 & 0 & 35 & 0\tabularnewline
pass\_engl & 0 & 1.00 & 1 & 1 & 0 & 2 & 0\tabularnewline
pass\_math & 0 & 1.00 & 1 & 1 & 0 & 2 & 0\tabularnewline
term\_year & 0 & 1.00 & 4 & 4 & 0 & 2 & 0\tabularnewline
\bottomrule
\end{longtable}

\textbf{Variable type: numeric}

\begin{longtable}[]{@{}lrrrrrrrrrl@{}}
\toprule
skim\_variable & n\_missing & complete\_rate & mean & sd & p0 & p25 &
p50 & p75 & p100 & hist\tabularnewline
\midrule
\endhead
hours\_attempted & 0 & 1 & 12.29 & 6.27 & 0 & 9.0 & 12.0 & 15.0 & 54 &
▆▇▁▁▁\tabularnewline
hours\_earned & 0 & 1 & 7.72 & 7.40 & 0 & 3.0 & 6.0 & 12.0 & 54 &
▇▃▁▁▁\tabularnewline
mc\_gpa & 0 & 1 & 2.16 & 1.48 & 0 & 0.5 & 2.5 & 3.5 & 4 &
▆▂▃▅▇\tabularnewline
\bottomrule
\end{longtable}

\hypertarget{missing-values-and-data-anomalies}{%
\subsection{Missing Values and data
Anomalies}\label{missing-values-and-data-anomalies}}

\begin{Shaded}
\begin{Highlighting}[]
\NormalTok{df_credits}\OperatorTok\StringTok{ }\KeywordTok{group_by}\NormalTok{(term_year,full_part)}\OperatorTok
\StringTok{                      }\KeywordTok{count}\NormalTok{(hours_attempted}\OperatorTok{==}\DecValTok{0}\NormalTok{)}
\end{Highlighting}
\end{Shaded}

\begin{verbatim}
## # A tibble: 7 x 4
## # Groups:   term_year, full_part [4]
##   term_year full_part `hours_attempted == 0`     n
##   <chr>     <chr>     <lgl>                  <int>
## 1 2020      FT        FALSE                   2252
## 2 2020      FT        TRUE                       3
## 3 2020      PT        FALSE                   1609
## 4 2020      PT        TRUE                      27
## 5 2021      FT        FALSE                   2147
## 6 2021      PT        FALSE                   1377
## 7 2021      PT        TRUE                      17
\end{verbatim}

Term year 2020: There are 3 full time and 27 part time students who
attempted zero hours. Term year 2021: There are 17 part time students
who attempted zero hours. These 47 observations will be dropped.

\begin{Shaded}
\begin{Highlighting}[]
\CommentTok{# remove students who attempted 0 hours}

\NormalTok{df_credits <-}\StringTok{ }\NormalTok{df_credits}\OperatorTok
\StringTok{              }\KeywordTok{filter}\NormalTok{(., hours_attempted }\OperatorTok{!=}\StringTok{ }\DecValTok{0}\NormalTok{)}
              
\KeywordTok{skim}\NormalTok{(df_credits)}
\end{Highlighting}
\end{Shaded}

\begin{longtable}[]{@{}ll@{}}
\caption{Data summary}\tabularnewline
\toprule
\endhead
Name & df\_credits\tabularnewline
Number of rows & 7385\tabularnewline
Number of columns & 18\tabularnewline
\_\_\_\_\_\_\_\_\_\_\_\_\_\_\_\_\_\_\_\_\_\_\_ &\tabularnewline
Column type frequency: &\tabularnewline
character & 15\tabularnewline
numeric & 3\tabularnewline
\_\_\_\_\_\_\_\_\_\_\_\_\_\_\_\_\_\_\_\_\_\_\_\_ &\tabularnewline
Group variables & None\tabularnewline
\bottomrule
\end{longtable}

\textbf{Variable type: character}

\begin{longtable}[]{@{}lrrrrrrr@{}}
\toprule
skim\_variable & n\_missing & complete\_rate & min & max & empty &
n\_unique & whitespace\tabularnewline
\midrule
\endhead
u\_number & 0 & 1.00 & 8 & 8 & 0 & 7385 & 0\tabularnewline
sex & 0 & 1.00 & 1 & 1 & 0 & 4 & 0\tabularnewline
race & 0 & 1.00 & 5 & 22 & 0 & 9 & 0\tabularnewline
age & 0 & 1.00 & 4 & 7 & 0 & 5 & 0\tabularnewline
high\_school & 0 & 1.00 & 7 & 30 & 0 & 169 & 0\tabularnewline
full\_part & 0 & 1.00 & 2 & 2 & 0 & 2 & 0\tabularnewline
city & 19 & 1.00 & 5 & 19 & 0 & 130 & 0\tabularnewline
stat\_code & 19 & 1.00 & 2 & 2 & 0 & 16 & 0\tabularnewline
zip & 19 & 1.00 & 4 & 5 & 0 & 164 & 0\tabularnewline
pell\_grant & 0 & 1.00 & 1 & 1 & 0 & 2 & 0\tabularnewline
camp\_code & 152 & 0.98 & 1 & 1 & 0 & 6 & 0\tabularnewline
major & 33 & 1.00 & 1 & 61 & 0 & 35 & 0\tabularnewline
pass\_engl & 0 & 1.00 & 1 & 1 & 0 & 2 & 0\tabularnewline
pass\_math & 0 & 1.00 & 1 & 1 & 0 & 2 & 0\tabularnewline
term\_year & 0 & 1.00 & 4 & 4 & 0 & 2 & 0\tabularnewline
\bottomrule
\end{longtable}

\textbf{Variable type: numeric}

\begin{longtable}[]{@{}lrrrrrrrrrl@{}}
\toprule
skim\_variable & n\_missing & complete\_rate & mean & sd & p0 & p25 &
p50 & p75 & p100 & hist\tabularnewline
\midrule
\endhead
hours\_attempted & 0 & 1 & 12.36 & 6.22 & 1 & 9.00 & 12.0 & 15.0 & 54 &
▆▇▁▁▁\tabularnewline
hours\_earned & 0 & 1 & 7.77 & 7.40 & 0 & 3.00 & 6.0 & 12.0 & 54 &
▇▃▁▁▁\tabularnewline
mc\_gpa & 0 & 1 & 2.17 & 1.48 & 0 & 0.62 & 2.5 & 3.5 & 4 &
▆▂▃▅▇\tabularnewline
\bottomrule
\end{longtable}

\hypertarget{degree-seeking-students}{%
\subsection{Degree Seeking Students}\label{degree-seeking-students}}

Create dataframe for degree seeking students only

\begin{Shaded}
\begin{Highlighting}[]
\CommentTok{#check majors list for undeclared and missing values.}

\NormalTok{df_credits}\OperatorTok\StringTok{ }\KeywordTok{count}\NormalTok{(major)}
\end{Highlighting}
\end{Shaded}

\begin{verbatim}
## # A tibble: 36 x 2
##    major                        n
##    <chr>                    <int>
##  1 0                           16
##  2 American Sign Language      12
##  3 Applied Geography           10
##  4 Architectural Technology    75
##  5 Art                        108
##  6 Broadcast Media             29
##  7 Building Trade              88
##  8 Business                   880
##  9 Certificate & LR           221
## 10 Communication               68
## # ... with 26 more rows
\end{verbatim}

\begin{Shaded}
\begin{Highlighting}[]
\KeywordTok{dim}\NormalTok{(df_credits)}
\end{Highlighting}
\end{Shaded}

\begin{verbatim}
## [1] 7385   18
\end{verbatim}

There are 72 students with undeclared majors. 16 students with majors
listed as 0 and 33 listed as NA. Due to privacy reasons, students have
been grouped so as not to identify individual students. Degree Major
groupings with less than 10 students have been listed as NA or 0. At
this point in time I will not remove these data points, because for the
purposes of this study, Major names are not required. I will change the
NA in major to `Field2' and the 0 to Field1. Students enrolled for a
certificate or Letter of Recommendation have been grouped togther
irrespective of field.

Create dataframe with degrees only

\begin{Shaded}
\begin{Highlighting}[]
\CommentTok{#remove certificates, NA refers to degree majors which have not been named to protect student privacy.}
\NormalTok{df_creditsDegrees<-}\KeywordTok{subset}\NormalTok{(df_credits, major}\OperatorTok{!=}\StringTok{"Certificate & LR"} \OperatorTok{||}\StringTok{ }\NormalTok{major }\OperatorTok{==}\StringTok{'NA'}\NormalTok{)}
\NormalTok{df_creditsDegrees}\OperatorTok\KeywordTok{count}\NormalTok{(major)}
\end{Highlighting}
\end{Shaded}

\begin{verbatim}
## # A tibble: 36 x 2
##    major                        n
##    <chr>                    <int>
##  1 0                           16
##  2 American Sign Language      12
##  3 Applied Geography           10
##  4 Architectural Technology    75
##  5 Art                        108
##  6 Broadcast Media             29
##  7 Building Trade              88
##  8 Business                   880
##  9 Certificate & LR           221
## 10 Communication               68
## # ... with 26 more rows
\end{verbatim}

There are 221 Certificate \& LR students. I will drop these non-degree
programmes and focus on degree majors. I have retained tha majors
grouped as NA.

\hypertarget{full-time-versus-part-time}{%
\subsection{Full time versus
Part-time}\label{full-time-versus-part-time}}

\begin{Shaded}
\begin{Highlighting}[]
\CommentTok{#count number of full time versus part-time degree students.}

\NormalTok{df_creditsDegrees}\OperatorTok\StringTok{ }\KeywordTok{group_by}\NormalTok{(term_year)}\OperatorTok
\StringTok{                     }\KeywordTok{count}\NormalTok{(full_part)}
\end{Highlighting}
\end{Shaded}

\begin{verbatim}
## # A tibble: 4 x 3
## # Groups:   term_year [2]
##   term_year full_part     n
##   <chr>     <chr>     <int>
## 1 2020      FT         2252
## 2 2020      PT         1609
## 3 2021      FT         2147
## 4 2021      PT         1377
\end{verbatim}

Degree Seeking students:In term year 2020, there were 2252 full time
degree seeking students and 1609 part time degree seeking students. In
term year 2021, there were 2147 full time degree seeking students and
1377 part time degree seeking students. Note that these numbers are
presented after the data has been partially cleaned. More rows of data
will be removed after the other variables of interest have been cleaned.

\hypertarget{hours-attempted-and-hours-earned-by-degree-seeking-students}{%
\subsection{Hours Attempted and Hours Earned by Degree Seeking
Students}\label{hours-attempted-and-hours-earned-by-degree-seeking-students}}

\begin{Shaded}
\begin{Highlighting}[]
\CommentTok{# AP credits can be given for courses taken at highschool.}
\CommentTok{#Check number of students with hours_earned greater than hours_attempted}

\NormalTok{df_credits }\OperatorTok\StringTok{ }\KeywordTok{count}\NormalTok{(hours_attempted}\OperatorTok{<}\NormalTok{hours_earned)}
\end{Highlighting}
\end{Shaded}

\begin{verbatim}
## # A tibble: 2 x 2
##   `hours_attempted < hours_earned`     n
##   <lgl>                            <int>
## 1 FALSE                             7380
## 2 TRUE                                 5
\end{verbatim}

There are 5 students with attempted hours smaller than hours earned.
This may be due to AP credits and will be addressed later.

Histogram of hours\_attempted by year

\begin{Shaded}
\begin{Highlighting}[]
\NormalTok{p <-}\StringTok{ }\KeywordTok{ggplot}\NormalTok{(df_creditsDegrees, }\KeywordTok{aes}\NormalTok{(}\DataTypeTok{x =}\NormalTok{ hours_attempted))}\OperatorTok{+}\StringTok{ }\KeywordTok{geom_histogram}\NormalTok{(}\KeywordTok{aes}\NormalTok{(}\DataTypeTok{fill =}\NormalTok{ term_year))}
\NormalTok{p1<-}\StringTok{ }\NormalTok{p }\OperatorTok{+}\StringTok{ }\KeywordTok{facet_wrap}\NormalTok{(}\OperatorTok{~}\NormalTok{full_part)}

\NormalTok{p1}
\end{Highlighting}
\end{Shaded}

\begin{verbatim}
## `stat_bin()` using `bins = 30`. Pick better value with `binwidth`.
\end{verbatim}

\includegraphics{PartBv5_files/figure-latex/unnamed-chunk-13-1.pdf}

Boxplots of hours\_attempted by year

\begin{Shaded}
\begin{Highlighting}[]
\NormalTok{p11 =}\StringTok{ }\KeywordTok{ggplot}\NormalTok{(df_creditsDegrees, }\KeywordTok{aes}\NormalTok{(hours_attempted))}
\NormalTok{p11 }\OperatorTok{+}\StringTok{ }\KeywordTok{geom_boxplot}\NormalTok{(}\KeywordTok{aes}\NormalTok{(}\DataTypeTok{colour =}\NormalTok{ term_year)) }\OperatorTok{+}
\StringTok{       }\KeywordTok{facet_wrap}\NormalTok{(}\OperatorTok{~}\NormalTok{full_part)}
\end{Highlighting}
\end{Shaded}

\includegraphics{PartBv5_files/figure-latex/unnamed-chunk-14-1.pdf}

Density plot of hours\_attempted by year

\begin{Shaded}
\begin{Highlighting}[]
\KeywordTok{ggplot}\NormalTok{(df_creditsDegrees, }\KeywordTok{aes}\NormalTok{(hours_attempted, }\DataTypeTok{fill =}\NormalTok{ term_year)) }\OperatorTok{+}\StringTok{ }\KeywordTok{geom_density}\NormalTok{(}\DataTypeTok{alpha =} \FloatTok{0.2}\NormalTok{) }\OperatorTok{+}
\StringTok{  }\KeywordTok{facet_wrap}\NormalTok{(}\OperatorTok{~}\NormalTok{full_part)}\OperatorTok{+}
\StringTok{  }\KeywordTok{xlab}\NormalTok{(}\StringTok{"Hours attempted"}\NormalTok{) }\OperatorTok{+}
\StringTok{  }\KeywordTok{ylab}\NormalTok{( }\StringTok{"Density"}\NormalTok{)}
\end{Highlighting}
\end{Shaded}

\includegraphics{PartBv5_files/figure-latex/unnamed-chunk-15-1.pdf}
Fivenum Summary

\begin{Shaded}
\begin{Highlighting}[]
\NormalTok{df_credits}\OperatorTok\StringTok{ }\KeywordTok{group_by}\NormalTok{(term_year,full_part)}\OperatorTok
\StringTok{  }\KeywordTok{summarise}\NormalTok{(}\DataTypeTok{n =} \KeywordTok{n}\NormalTok{(),}
            \DataTypeTok{min =} \KeywordTok{fivenum}\NormalTok{(hours_attempted)[}\DecValTok{1}\NormalTok{],}
            \DataTypeTok{Q1 =} \KeywordTok{fivenum}\NormalTok{(hours_attempted)[}\DecValTok{2}\NormalTok{],}
            \DataTypeTok{median =} \KeywordTok{fivenum}\NormalTok{(hours_attempted)[}\DecValTok{3}\NormalTok{],}
            \DataTypeTok{Q3 =} \KeywordTok{fivenum}\NormalTok{(hours_attempted)[}\DecValTok{4}\NormalTok{],}
            \DataTypeTok{max =} \KeywordTok{fivenum}\NormalTok{(hours_attempted)[}\DecValTok{5}\NormalTok{])}
\end{Highlighting}
\end{Shaded}

\begin{verbatim}
## `summarise()` regrouping output by 'term_year' (override with `.groups` argument)
\end{verbatim}

\begin{verbatim}
## # A tibble: 4 x 8
## # Groups:   term_year [2]
##   term_year full_part     n   min    Q1 median    Q3   max
##   <chr>     <chr>     <int> <dbl> <dbl>  <dbl> <dbl> <dbl>
## 1 2020      FT         2252     3    12     13    15    53
## 2 2020      PT         1609     1     6      8    10    51
## 3 2021      FT         2147     3    12     14    16    54
## 4 2021      PT         1377     1     5      8    10    47
\end{verbatim}

Students who attempted less than 3 hours

\begin{Shaded}
\begin{Highlighting}[]
\NormalTok{df_creditsDegrees}\OperatorTok\StringTok{ }\KeywordTok{group_by}\NormalTok{(term_year,full_part)}\OperatorTok
\StringTok{                      }\KeywordTok{count}\NormalTok{(hours_attempted}\OperatorTok{<}\DecValTok{3}\NormalTok{)}
\end{Highlighting}
\end{Shaded}

\begin{verbatim}
## # A tibble: 6 x 4
## # Groups:   term_year, full_part [4]
##   term_year full_part `hours_attempted < 3`     n
##   <chr>     <chr>     <lgl>                 <int>
## 1 2020      FT        FALSE                  2252
## 2 2020      PT        FALSE                  1568
## 3 2020      PT        TRUE                     41
## 4 2021      FT        FALSE                  2147
## 5 2021      PT        FALSE                  1361
## 6 2021      PT        TRUE                     16
\end{verbatim}

Students who attempted more than 18hours

\begin{Shaded}
\begin{Highlighting}[]
\NormalTok{df_creditsDegrees}\OperatorTok\StringTok{ }\KeywordTok{group_by}\NormalTok{(term_year,full_part)}\OperatorTok
\StringTok{                      }\KeywordTok{count}\NormalTok{(hours_attempted}\OperatorTok{>}\DecValTok{18}\NormalTok{)}
\end{Highlighting}
\end{Shaded}

\begin{verbatim}
## # A tibble: 8 x 4
## # Groups:   term_year, full_part [4]
##   term_year full_part `hours_attempted > 18`     n
##   <chr>     <chr>     <lgl>                  <int>
## 1 2020      FT        FALSE                   1925
## 2 2020      FT        TRUE                     327
## 3 2020      PT        FALSE                   1553
## 4 2020      PT        TRUE                      56
## 5 2021      FT        FALSE                   1857
## 6 2021      FT        TRUE                     290
## 7 2021      PT        FALSE                   1330
## 8 2021      PT        TRUE                      47
\end{verbatim}

\hypertarget{hours-attempted-summary}{%
\subsubsection{Hours Attempted Summary}\label{hours-attempted-summary}}

The distributions appeared to be skewed to the right. Incoming students
are allowed to get AP credits for courses taken at highschool. The
distributions for part-time student in 2019 and 2020 appear to follow a
similar distribution. Peaks can be attributed to the fact that most
courses are 3 or 4 credit courses. The college offers very few 2 credit
courses. 41 part time students attempted less than 3 hours in term year
2020 and 16 part time students attempted less than 3 hours in term year
2021. Students also need special permission to enroll in more than 18
credits. In term year 2020, 327 full time and 56 part time students
attempted more than 18 hours. In term year 2021, 290 full time and 47
part time degree seeking students attempted more than 18 hours. As non
credit hours and credit hours are included in the hours attempted
variable, I will use the hours earned variable to filter the datasets
and remove outliers.

Histogram of hours\_earned by year

\begin{Shaded}
\begin{Highlighting}[]
\NormalTok{p <-}\StringTok{ }\KeywordTok{ggplot}\NormalTok{(df_creditsDegrees, }\KeywordTok{aes}\NormalTok{(}\DataTypeTok{x =}\NormalTok{ hours_earned))}\OperatorTok{+}\StringTok{ }\KeywordTok{geom_histogram}\NormalTok{(}\KeywordTok{aes}\NormalTok{(}\DataTypeTok{fill =}\NormalTok{ term_year))}
\NormalTok{p1<-}\StringTok{ }\NormalTok{p }\OperatorTok{+}\StringTok{ }\KeywordTok{facet_wrap}\NormalTok{(}\OperatorTok{~}\NormalTok{full_part)}

\NormalTok{p1}
\end{Highlighting}
\end{Shaded}

\begin{verbatim}
## `stat_bin()` using `bins = 30`. Pick better value with `binwidth`.
\end{verbatim}

\includegraphics{PartBv5_files/figure-latex/unnamed-chunk-19-1.pdf}
Boxplots of hours\_earned by year

\begin{Shaded}
\begin{Highlighting}[]
\NormalTok{p11 =}\StringTok{ }\KeywordTok{ggplot}\NormalTok{(df_creditsDegrees, }\KeywordTok{aes}\NormalTok{(hours_earned))}
\NormalTok{p11 }\OperatorTok{+}\StringTok{ }\KeywordTok{geom_boxplot}\NormalTok{(}\KeywordTok{aes}\NormalTok{(}\DataTypeTok{colour =}\NormalTok{ term_year)) }\OperatorTok{+}
\StringTok{       }\KeywordTok{facet_wrap}\NormalTok{(}\OperatorTok{~}\NormalTok{full_part)}
\end{Highlighting}
\end{Shaded}

\includegraphics{PartBv5_files/figure-latex/unnamed-chunk-20-1.pdf}

Density plot of hours\_earned by year

\begin{Shaded}
\begin{Highlighting}[]
\KeywordTok{ggplot}\NormalTok{(df_credits, }\KeywordTok{aes}\NormalTok{(hours_earned, }\DataTypeTok{fill =}\NormalTok{ term_year)) }\OperatorTok{+}\StringTok{ }\KeywordTok{geom_density}\NormalTok{(}\DataTypeTok{alpha =} \FloatTok{0.2}\NormalTok{) }\OperatorTok{+}
\StringTok{  }\KeywordTok{facet_wrap}\NormalTok{(}\OperatorTok{~}\NormalTok{full_part)}\OperatorTok{+}
\StringTok{  }\KeywordTok{xlab}\NormalTok{(}\StringTok{"Hours Earned"}\NormalTok{) }\OperatorTok{+}
\StringTok{  }\KeywordTok{ylab}\NormalTok{( }\StringTok{"Density"}\NormalTok{)}
\end{Highlighting}
\end{Shaded}

\includegraphics{PartBv5_files/figure-latex/unnamed-chunk-21-1.pdf}
Hours Earned Fivenum Summary

\begin{Shaded}
\begin{Highlighting}[]
\NormalTok{df_credits}\OperatorTok\StringTok{ }\KeywordTok{group_by}\NormalTok{(term_year,full_part)}\OperatorTok
\StringTok{  }\KeywordTok{summarise}\NormalTok{(}\DataTypeTok{n =} \KeywordTok{n}\NormalTok{(),}
            \DataTypeTok{min =} \KeywordTok{fivenum}\NormalTok{(hours_earned)[}\DecValTok{1}\NormalTok{],}
            \DataTypeTok{Q1 =} \KeywordTok{fivenum}\NormalTok{(hours_earned)[}\DecValTok{2}\NormalTok{],}
            \DataTypeTok{median =} \KeywordTok{fivenum}\NormalTok{(hours_earned)[}\DecValTok{3}\NormalTok{],}
            \DataTypeTok{Q3 =} \KeywordTok{fivenum}\NormalTok{(hours_earned)[}\DecValTok{4}\NormalTok{],}
            \DataTypeTok{max =} \KeywordTok{fivenum}\NormalTok{(hours_earned)[}\DecValTok{5}\NormalTok{])}
\end{Highlighting}
\end{Shaded}

\begin{verbatim}
## `summarise()` regrouping output by 'term_year' (override with `.groups` argument)
\end{verbatim}

\begin{verbatim}
## # A tibble: 4 x 8
## # Groups:   term_year [2]
##   term_year full_part     n   min    Q1 median    Q3   max
##   <chr>     <chr>     <int> <dbl> <dbl>  <dbl> <dbl> <dbl>
## 1 2020      FT         2252     0     6      9    13    53
## 2 2020      PT         1609     0     0      3     6    51
## 3 2021      FT         2147     0     6     10    13    54
## 4 2021      PT         1377     0     0      3     6    47
\end{verbatim}

\hypertarget{hours-earned-summary}{%
\subsubsection{Hours Earned Summary}\label{hours-earned-summary}}

The distributions appeared to be skewed to the right. Incoming students
are allowed to get AP credits for courses taken at highschool. The
distributions for part-time student in 2019 and 2020 appear to follow a
similar distribution. Peaks can be attributed to the fact that most
courses are 3 or 4 credit courses. The college offers very few 2 credit
courses. Students also need special permission to enroll in more than
18credits.

Students who earned less than 3 hours

\begin{Shaded}
\begin{Highlighting}[]
\NormalTok{df_creditsDegrees}\OperatorTok\StringTok{ }\KeywordTok{group_by}\NormalTok{(term_year,full_part)}\OperatorTok
\StringTok{                      }\KeywordTok{count}\NormalTok{(hours_earned}\OperatorTok{<}\DecValTok{3}\NormalTok{)}
\end{Highlighting}
\end{Shaded}

\begin{verbatim}
## # A tibble: 8 x 4
## # Groups:   term_year, full_part [4]
##   term_year full_part `hours_earned < 3`     n
##   <chr>     <chr>     <lgl>              <int>
## 1 2020      FT        FALSE               2012
## 2 2020      FT        TRUE                 240
## 3 2020      PT        FALSE                882
## 4 2020      PT        TRUE                 727
## 5 2021      FT        FALSE               1857
## 6 2021      FT        TRUE                 290
## 7 2021      PT        FALSE                873
## 8 2021      PT        TRUE                 504
\end{verbatim}

Hours earned equate to credits earned and is the indication of the
progress a student is making towards a degree. In term year 2020, 240
full time and 727 part time students earned less than 3 credits. In term
year 2021, 290 full time and 504 part time earned less than 3 credits.

Students who earned more than 18 credits

\begin{Shaded}
\begin{Highlighting}[]
\NormalTok{df_creditsDegrees}\OperatorTok\StringTok{ }\KeywordTok{group_by}\NormalTok{(term_year,full_part)}\OperatorTok
\StringTok{                      }\KeywordTok{count}\NormalTok{(hours_earned}\OperatorTok{>}\DecValTok{18}\NormalTok{)}
\end{Highlighting}
\end{Shaded}

\begin{verbatim}
## # A tibble: 8 x 4
## # Groups:   term_year, full_part [4]
##   term_year full_part `hours_earned > 18`     n
##   <chr>     <chr>     <lgl>               <int>
## 1 2020      FT        FALSE                2015
## 2 2020      FT        TRUE                  237
## 3 2020      PT        FALSE                1570
## 4 2020      PT        TRUE                   39
## 5 2021      FT        FALSE                1955
## 6 2021      FT        TRUE                  192
## 7 2021      PT        FALSE                1337
## 8 2021      PT        TRUE                   40
\end{verbatim}

EDA Questions: Will only use students seeking a degree.

\begin{enumerate}
\def\labelenumi{\arabic{enumi}.}
\tightlist
\item
  Difference in enrollement numbers, demographics (race, gender,
  highschool) in term year 2021 and term year 2020.
\item
  Did students in term year 2021 attempt more or less hours than
  students in term year 2020.
\item
  Did students in term year 2021 earn more or less hours than students
  in term year 2020.
\item
  Who are the students in the tail end of the hours attempted
  distributions. ( attempted less than 3 hours or attempted more than
  18hours)
\item
  Who are the students in the tail end of the hours earned
  distributions. (Earn more than 18 credits or earn less than 3 credits)
\item
  Which subgroup is the best predictor of the population. Subgroup most
  like the population. ( Smallest Effect Size.)
\item
  Distribution of hours\_attempted - hours earned.
\item
  Rate at which students earn credit.
\end{enumerate}

\end{document}
